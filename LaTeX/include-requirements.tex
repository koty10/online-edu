
%!TEX ROOT=ctutest.tex

\chapter{Obecné požadavky na systém}

V této první kapitole jsem se zaměřil na souhrn hlavních požadavků. V první řadě se jedná o obecné požadavky zejména z uživatelského hlediska, ale v druhé polovině jsem směřoval pozornost i na technickou stránku věci.
Většina bodů vzešla z rozhovorů s učiteli. Tedy s těmi, kdo budou systém eventuálně využívat primárně.











\section{Jednoduchost a přehlednost}

Hlavním požadavkem je, aby aplikace byla co nejjednodušší, nejpřehlednější, ale aby zároveň poskytovala všechny potřebné funkce.



\section{Estetické zpracování}

Jelikož se jedná o produkt, který budou používat hlavně děti na základní škole, mělo by být přihlíženo na design. Zejména důležité jsou barvy, které by měly být pestré. Neméně významné jsou ale i obrázky. Žáky musí aplikace bavit a líbit se jim. Mimo jiné jim to dodá i tu správnou motivaci se do systému přihlašovat a aktivně plnit své úkoly.


\section{Individuální prostředí podle role uživatele}

Tento bod souvisí s tím předchozím. Není totiž samozřejmě nutné dbát tolik na estetiku, tím myslím hlavně obrázky, například u učitele. Ten by měl spíše mít podrobnější přehled o třídě a o žácích.

To samé platí pro rodiče. Mnohem důležitější je pro ně vidět přehled o svém dítěti včetně různých statistik. Na druhou stranu i pro tyto uživatele musí zůstat systém co nejvíce přehledný a jednoduchý. Ne každý musí být totiž tolik technicky zdatný, aby si poradil se složitějším prostředím.



\section{Odolnost proti nechtěným akcím}

Snaha by také měla být o to, aby systém v maximální míře chránil uživatele proti nechtěným akcím. Například by se nemělo stávat, že někdo něco smaže omylem jen proto, že mu to systém jednoduše dovolí. Všechny tyto akce by měly být potvrzovány například dialogem.



\section{Spolehlivost}

Alfou a omegou úspěchu této aplikace je ale v konečném důsledku hlavně její spolehlivost. Není přípustné, aby docházelo k chybám typu, že něco zmizí nebo nebude dostupné. 

Uživatel musí mít jistotu, že vždy najde to, co do systému vložil. Formuláře nesmí dovolit vložení nepřípustných dat a měly by na to i upozornit. 



\section{Bezpečnost}

V takovémto typu software je také bezpozmínečně nutné nějakým způsobem řešit bezpečnost.

Aby se uživatel vůbec k něčemu dostal, musí se neprve přihlásit. To znamená, že zadá jméno, heslo a pokud bude úspěšně autenzizován, tak musí být následně autorizován k přístupu na jednotlivé stránky

Jednotlivé role mají přístup pouze k určité části aplikace. Například student nemůže zadávat úkoly.